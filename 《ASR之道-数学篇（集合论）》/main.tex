\documentclass{article}
\usepackage[utf8]{inputenc}
\usepackage{xeCJK}
\usepackage{bm}
\usepackage{amsmath,amssymb}


\title{ASR Principia 《ASR之道-数学篇(集合论)》}
\author{xiao11lam/ZHANG XIAO }
\date{June 2020}

\usepackage{natbib}
\usepackage{graphicx}

\begin{document}

\maketitle
\begin{abstract}
在开篇中,笔者提到要从道(原理)和术(应用)这两个大的层次框架来介绍语音识别技术。本部分主要是围绕道这一主线,但是在正式介绍语音识别技术之前,我们将谈到数学,声学,信号处理,机器学习和语言学相关的背景知识。语音识别技术涉及的领域众多,如果不能博采众长,那么相关的学习和学术也就难以进行。尤其是数学的运用,所以把它放在首要位置。在ASR中主要用到的数学分支有:离散/组合数学中的图论,统计概率,数学分析(包括实分析,复分析以及调和分析/傅里叶分析)以及最优化。本书与其他handbook不同的是很多会从严谨的数学定义出发,而是着重于快速运用和开发。对本身其来龙去脉不会过多着笔,而是追求一种高屋建瓴的思维模式,从整体中管中窥豹。文章末尾也会提供相关文献专著以供有兴趣的读者深入学习。如此一来,“今天格一件,明天格一件,习而久之,自然贯通”。考虑到语音识别相关领域大部分文献以英文为主,为了让读者更好地回顾,在正文部分将主要用英文来进行介绍。\\
笔者希望读者对集合已有了解,此文仅仅是对其概念最为简单的回顾,想要深入了解的话请查看推荐书录。
\end{abstract}

\section{Sets}
A \bm{$set$} is any well-defined list or collection of objects. The objects in a set are called \bm{$elements$}, or  \bm{$members$}.
\begin{flusleft}
Like A = \left\{a_{1}, a_{2}, a_{3}\right\} 
\quad 
     B = \left\{b_{1}, b_{2}, b_{3}\right\}
\squad
     C = \left\{a_{1}, a_{2}\right\} 
\end{flusleft}
\subsection{Subsets}
We say C is a subset of A, and write: C\subseteq{$A$}\\
\begin{flushleft}
Since C is a subset of A ,and C not euqal A, so we call A is a proper subset of B, we say A \subset B.
\end{flushleft}

\subsection{Power Set}
The set of all subsets of a set A is called the \bm{$power \ set$} of A, and is denoted by P(A).\\
Eg. \squad P(A) = \left\{\varnothing,
\left\{a_{1}\right\}, 
\left\{a_{2}\right\},
\left\{a_{3}\right\},
\left\{a_{1}a_{2}\right\},
\left\{a_{1}a_{3}\right\},
\left\{a_{2}a_{3}\right\},
A
\right\}
\subsection{Operations on sets}
\begin{itemize}
    \item  Union(交)
\begin{flushleft}
\cap \\
\end{flushleft}
    \item  Intersections(并)
\begin{flushleft}
\cup \\
\end{flushleft}
    \item  Complements(补)
\begin{flushleft}
like A^{'}, B^{'}, C^{'}
\end{flushleft}
\end{itemize}

\subsection{Sets of numbers}
Here are some important default sets:\\

\bm{$N$}: the set of all \bm{$nature\ numbers$}\\
\bm{$Z$}: the set of all \bm{$integers$}\\
\bm{$Q$}: the set of all \bm{$rational\ numbers$}\\
\bm{$Q^{'}$}: the set of all \bm{$irrational\ numbers$}\\
\bm{$R$}: the set of all \bm{$real\ numbers$}\\
\section{Relation}
\subsection{Product set or Cartesian product}
the formuala is:
A \times B = \left\{\left(a,b\right)\mid a\in A \ and \ b\in B\right\}

\begin{flushleft}
So it means: A \times B = \left\{\left(a_{1},b_{1}\right),
       \left(a_{1},b_{2}\right),
       \left(a_{1},b_{3}\right),
       \left(a_{2},b_{1}\right),
       \left(a_{2},b_{2}\right),
       \left(a_{2},b_{3}\right),
       \left(a_{3},b_{1}\right),
       \left(a_{3},b_{2}\right),
       \left(a_{3},b_{3}\right)
\right\}
\end{flushleft}

\begin{flushleft}
\qquad \qquad A^2 = A \times A = \left\{\left(
            a_{1},a_{1}\right),
       \left(a_{1},a_{2}\right),
       \left(a_{1},a_{3}\right),
       \left(a_{2},a_{1}\right),
       \left(a_{2},a_{2}\right),
       \left(a_{2},a_{3}\right),
       \left(a_{3},a_{1}\right),
       \left(a_{3},a_{2}\right),
       \left(a_{3},a_{3}\right)
\right\}
\end{flushleft}

\subsection{Binary Relations}
A binary relation \bm{$R$} from A to B is \bm{$a \ subset \of\ A \times B$} \bm{$(R\subseteq$A\times B$)$}
\\
\\
\bm{$aRb$} means: a is R-related to b
\\
\\
\bm{$aRb$} \squad \leftrightarrow\squad \left(a,b\right)\in \bm{$R$}


\subsection{Domain and Range of R}

\bm{domain}: Dom(R) = \left\{a\in A\mid  aRb  \right\}
\begin{flushleft}
Domain is the set of all a values of (a,b) in R
\end{flushleft}
\bm{range}: Ran(R) = \left\{b\in B\mid aRb \right\}
\begin{flushleft}
Range is the set of all b values of (a,b) in R
\end{flushleft}


\subsection{Matrix of a relation}
The \bm{$matrix$} of R is denoted  \bm{M_{R}} = \left[m_{i}_{j}\right]\\
In here:\\ 
if a_{i}Rb_{j}, m_{i}_{j}=1\\
if a_{i}\not{$R$}b_{j}, m_{i}{j}=0\\
From the previous conditions, we know R is a subset of A\timesB, so R can be randomly chosen to describe the correlations between two sets. This technique is always used even in neuron network and deep learning and later we will see that. \\
We can use the matrix of R to represent the relationship.\\
Example: we now pick out a R we want, let's say R = {\left(
            a_{1},b_{3}\right),
       \left(a_{2},b_{2}\right),
       \left(a_{3},b_{1}\right)
\right\}\\

The original correlation matrix = \begin{bmatrix} a_{1}b_{1} & a_{1}b_{2} & a_{1}b_{3} \\ a_{2}b{1} & a_{2}b{2} & a_{2}{3}\\
a_{3}b{1} & a_{3}b{2} & a_{3}{3}\end{bmatrix}\\
After that we using R to check, judge whether the elements in R still in that matrix, if in then change into 1, if not, then change into 0.\\
Now, the R = M_{R}=\begin{bmatrix} 0 & 0 & 1 \\ 0 & 1 & 0\\
1 & 0 & 0\end{bmatrix}\\
\subsection{Properties of Relations}
\begin{itemize}
    \item Reflexive(自反关系)
    \begin{center}
        R is Reflexive if aRa, ${\forall a}$\in A\\
        {$aRa$} \squad \leftrightarrow\squad \left(a,a\right)\in R
    \end{center}
    
    \item Symmetric(对称关系)
    \begin{center}
    R is Symmetric if aRb, ${\forall a,b}$\in A\\ 
    {$aRb$}\rightarrow{$bRa$}
    \\
    \begin{center}
    \left(a,b\right)\in{$R$}\rightarrow{$\left(b,a\right)$}\in{$R$}
    \end{center}
    \end{center}
    
    \item Transitive(转换关系)
    \begin{center}
    R is Transitive if aRb and bRc, ${\forall a,b,c}$\in A\\
    \begin{center}
    aRb and bRc \rightarrow{aRc}\\
    \left(a,b\right)\in{$R$} \cap  \left(b,c\right)\in{$R$}  \rightarrow{$\left(a,c\right)$\in{$R$}
    \end{center}
    \end{center}
    \\
    \item Equivalence(相等关系)\\
    If R is reflexive,symmetric and transitive, then we call R is an equivalence equation.\\
\begin{equvilence}
\left\{
    \begin{array}{lr}
    aRa\leftrightarrow\squad\left(a,a\right)\in R & \\
    \left(a,b\right)\in{$R$}\rightarrow{$\left(b,a\right)$\in {$R$}} \\
    \left(a,b\right)\in{$R$} \cap  \left(b,c\right)\in{$R$}  \rightarrow{$\left(a,c\right)$\in{$R$}, &
    \end{array}
\right.
\end{equvilence}
\end{itemize}






\section{Suggested Materials(推荐书目)}
Basic Set Theory --- Azriel Lévy\\
Set Theory: An Introduction to Independence Proofs --- Kenneth Kunen\\
Set Theory: An Introduction to Large Cardinals (Studies in Logic & the Foundations of Mathematics - Vol 76) --- Frank R Drake \\
Set Theory --- Jech, Thomas\\

\begin{flushleft}
最近手边增加了不少琐事,正所谓悠闲才是哲学之母,看来伊终究是与我无缘了。昨天拜会了Reuben Samuel先生,以后向他学习音乐的混音和录制,考虑到网上的相关学习分享可谓乏陈,待学成以后也能在这里单独开篇分享。ASR在工业界中在声乐处理中还是大有可为的,希望在音乐唱片录制的学习过程中有所广益。但是考虑到我的五音不全,又想到鹤善舞而不善耕,牛善耕而不善舞,物性然也。如今欲反而教之,无乃劳矣。\\
\end{flushleft}

\end{document}
